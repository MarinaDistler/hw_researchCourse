\documentclass[12pt]{article}
\usepackage[russian, english]{babel}
\usepackage[utf8]{inputenc}
\usepackage[T2A]{fontenc}
\usepackage[unicode, pdftex]{hyperref}
\usepackage{amsmath,amsfonts,amssymb}
\usepackage{graphicx}
\usepackage[unicode, pdftex]{hyperref}
\usepackage{a4wide}
\title{Comparative analysis of my projects}
%\author{not specified}
\date{}
\begin{document}
\maketitle

\section{Большие языковые модели (LLM) как универсальный интерфейс}
Предлагается использовать LLM для создания единого интерфейсая для нескольких обученных моделей. 
В качестве примера мы рассмотрим несколько промышленных задач регрессии и классификации, для каждой из которых будет своя обученная модель. Пользователь будет текстом описывать каккую задачу он хочет решить и передавать входные параметры. LLM должна будет понять какую модель выбрать и какие параметры подать на вход, чтобы получить прогноз.Также мы сравним насколько LLM самостоятельно, без обученных моделей, сможет решить поставленные задачи. 
 
\begin{enumerate}
\item \emph{The impact} Данный проект поможет понять насколько LLM могут быть полезны в решении узких промышленных задач
\item \emph{The consistency} Судя по описанию, доказательство через проведение эксперимента
\item \emph{The novelty} Использование LLM как надстройки над другими предобученными моделями
\item \emph{My contribution} Изучение альтернатив, проведение эксперимента по использованию LLM как интерфейса над другими обученными моделями
\item \emph{The project focuses}  В зависимости от результата, можно ли использовать LLM как интерфейс над другими предобученными моделями или она лучше решает задачи самостоятельно 
\end{enumerate}

\section{Исследование возможности улучшения качества \\генеративных моделей путем использования\\суррогатных признаков}
Проблема: при генерации бывает необходимо, чтобы воспроизводились определенные свойства, которыми обладают объекты по отдельности, т.е. которые могут быть вычисленные на уровне каждого примера. Не всегда данные метрики дифференцируемы, поэтому бывает невозможно использовать их в качестве функций потерь напрямую. Если же подавать их дискриминатору напрямую, то генератору может быть трудно воспроизводить их, по той же самой причине. 
\begin{enumerate}
\item \emph{The impact} Улучшение качества воспроизведения определенных свойства с помощью генеративных моделей
\item \emph{The consistency} По описанию не очень понятно, но скорее всего проведение эксперимента
\item \emph{The novelty} Использование суррогатных признаков в генеративных моделях для задания определенных свойств
\item \emph{My contribution} Изучение альтернатив, проведение эксперимента
\item \emph{The project focuses} В зависимости от результата, могут ли суррогатные признаки улучшить воспроизведение определенных свойств генеративными моделями
\end{enumerate}

\section{Reproducibility of AI models  (Воспроизводимость предсказаний моделей ИИ)}
Imagine two binary classifiers, achieving 95\% accuracy on the same data. Despite this high quality, the two models may disagree on 10\% of all observations, which would not be acceptable in many critical applications. The goals is to study the "disagreement" metric and its numerical properties
\begin{enumerate}
\item \emph{The impact} Улучшение возможности сравнения моделей с точки зрения того, где они ошибаются
\item \emph{The consistency} В зависимости от разработанной метрики
\item \emph{The novelty} В зависимости от разработанной метрики
\item \emph{My contribution} Изучение существующих метрик, сравнение, разработка новой
\item \emph{The project focuses} сравнении метрик воспроизводимости моделей и новой метрики
\end{enumerate}

\section{Resume}
Проект \emph{Большие языковые модели (LLM) как универсальный интерфейс} имеет наибольший приоритет, так как я хочу лучше разобраться в устройстве LLM и он в принципе достаточно подходит под требования, чтобы его можно было выбрать. Проект 3 наиболее подходит под требования для научной статьи, но там более высокий риск, что можно ничего не придумать хорошего, а улучшения качества как в первых 2 вариантах скорее всего даст результат. 

\end{document}