\documentclass[12pt]{article}
\usepackage[russian, english]{babel}
\usepackage[utf8]{inputenc}
\usepackage[T2A]{fontenc}
\usepackage[unicode, pdftex]{hyperref}
\usepackage{amsmath,amsfonts,amssymb}
\usepackage{graphicx}
\usepackage[unicode, pdftex]{hyperref}
\usepackage{a4wide}

\title{Преобразование прозы в стихотворную форму с помощью методов обработки текста}
%\author{not specified, not necessary here}
\date{}
\begin{document}
\maketitle
Примечание: к сожалению, у меня не получилось договорится о выполнении проекта о котором я писала в предыдущем домашнем задании, в итоге я выбрала проект по теме "Преобразование прозы в стихотворную форму с помощью методов обработки текста".

Разработать модель для преобразования прозы в стихотворную форму. Изучить существующие методы и проверить их эффективность для русского языка. Предложить и разработать улучшения для наилучшего метода.



\section{Introduction}
В таблице~\ref{tab:intro_comparative} описаны уже существующие решения для генерации стихов по прозаическому тексту. 

\begin{table}[!htbp]
\label{tab:intro_comparative}
\caption{Сравнительный анализ решений преобразования прозы в стихотворную форму.}
\begin{tabular}{p{5cm}|p{5cm}|p{5cm}}
	Solution & Strengths & Weakness \\
	\hline
	Automatic Poetry Generation from Prosaic Text~\cite{van-de-cruys-2020-automatic} & 
        Решает именно ту задачу, что мне нужна
        
        Не требует стихов для обучения
        &
	Не работает для русского языка 
        
        Уже несколько устарела (использует RNN)\\
        \hline
        System Supporting Poetry Generation Using Text Generation and Style Transfer Methods~\cite{styletranferBadura} &
        Использует трансформер-архитектуру
        
        Предлагает метрики оценивания сгенерированных стихов &
        Генерирует стих по первой строке, а не переводит прозу в стихотворную форму
        
        Не работает для русского языка\\
        \hline
        Zero-shot Sonnet Generation with Discourse-level Planning and Aesthetics Features~\cite{Zero-shot-Sonnet-Generation} &
        Не использует стихи для обучения
        
        Предлагает способ увеличить креативность модели 
        
        Предлагает автоматические метрики оценивания качества&
        
        Генерирует стих, а не переводит прозу в стихотворную форму
        
        Не работает для русского языка
\end{tabular} 
\end{table}
 
 The section references contain the full list, collected for this project. 
\nocite{*} % Remove this to keep the cited referernces only

\bibliographystyle{unsrt}
\bibliography{Distler-Step-7}
\end{document}