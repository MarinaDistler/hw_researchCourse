\documentclass[12pt]{article}
\usepackage[utf8]{inputenc}
\usepackage[T1]{fontenc}
\usepackage{amsmath,amsfonts,amssymb}
\usepackage{graphicx}
\usepackage{a4wide}
\title{Reconstructed abstract of the paper ``How much does it help to know what she knows you know? An agent-based simulation study''}
%\author{not specified, not necessary here}
\date{}
\begin{document}
\maketitle

\begin{abstract}

In psychology, the theory of mind is the ability to predict others' actions by attributing to them mental content like beliefs, and intentions. This ability can be recursive, with humans reasoning about how others reason. We use agent-based models to demonstrate that higher-order theory of mind offers an advantage in competitive settings. The advantage is consistent across four games, but its effectiveness decreases with the order of the theory of mind: first and second-order agents outperforme limited opponents, while more than third-order agents prove beneficial only in specific cases.
\end{abstract}
\paragraph{Keywords:} Agent-based computational models, theory of mind, strictly competitive games

\paragraph{Highlights:}
\begin{enumerate}
\item In some competitive settings higher-order theory of mind is profitable for agents
\item Across four discussed games there is a common pattern
\item The effect decreases with an increase in the order of the theory of mind
\item First and second-order agents outperform limited opponents
\item More than third-order agents prove beneficial only in specific cases
\end{enumerate}

\section{Introduction}
I chose the paper~\cite{DEWEERD201367}, because it describes important topic of using theory of mind in agent-based models. 

\bibliographystyle{unsrt}
\bibliography{Name-theArt}
\end{document}