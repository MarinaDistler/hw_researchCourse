\documentclass[12pt]{article}
\usepackage[russian, english]{babel}
\usepackage[utf8]{inputenc}
\usepackage[T2A]{fontenc}
\usepackage[unicode, pdftex]{hyperref}
\usepackage{amsmath,amsfonts,amssymb}
\usepackage{graphicx}
\usepackage[unicode, pdftex]{hyperref}
\usepackage{a4wide}

\begin{document}

\paragraph{Title:} Преобразование прозы в стихотворную форму с помощью методов обработки текста

\paragraph{Abstract:} 
Разработать модель для преобразования прозы в стихотворную форму. Изучить существующие методы и проверить их эффективность для русского языка. Предложить и разработать улучшения для наилучшего метода.

\paragraph{Datasets:} 
\begin{enumerate}
\item Поэтический корпус русского языка по \href{https://github.com/IlyaGusev/PoetryCorpus}{ссылке}.
\item Датасет размеченных английских стихов по \href{https://github.com/potamides/uniformers?tab=readme-ov-file}{ссылке} из статьи~\cite{belouadi-eger-2023-bygpt5}
\item Poetry Foundation Poems, датасет стихов с \href{https://www.poetryfoundation.org/}{сайта} по  \href{https://www.kaggle.com/datasets/tgdivy/poetry-foundation-poems}{ссылка}. 
\end{enumerate}

\paragraph{References:}  Papers with a fast intro and the basic solution to compare.
\begin{enumerate}
\item Формулировка проблемы преобразования текста в стихи, однако только для Персидского~\cite{Prose2PoemPersian}.
\item Базовый алгоритм - LLM-based multi-agent poetry generation in non-cooperative environments ~\cite{zhang2024llmbasedmultiagentpoetrygeneration}.
\end{enumerate}

\paragraph{Basic solution:} \href{https://github.com/zhangr2021/Multiagent_poetry}{Ссылка} на код базового алгоритма из статьи~\cite{zhang2024llmbasedmultiagentpoetrygeneration}. 

\paragraph{Authors:} М. Дистлер, В. Малых


\bibliographystyle{unsrt}
\bibliography{Name-theArt}
\end{document}