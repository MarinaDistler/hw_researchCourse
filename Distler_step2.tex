\documentclass[12pt]{article}
\usepackage[utf8]{inputenc}
\usepackage[T1]{fontenc}
\usepackage{amsmath,amsfonts,amssymb}
\usepackage{graphicx}
\usepackage{a4wide}\title{Industrial project description (put the project title here)}
%\author{not specified}
\date{}
\begin{document}
\maketitle

%\begin{abstract}
Answer the question to outline your project. Choose one of the roles: an {Expert} or an~\textbf{Analyst}.
%\end{abstract}
% \paragraph{Keywords:} The Art On Scientific Research, Abstract Reconstruction, Please Put Yours 


\section{Planning the industrial research project}

The goal of this project is to develop an accurate customer demand forecasting model using machine learning techniques. The expected outcome is a tool that leverages historical sales and customer interaction data to predict future demand, thereby aiding in efficient inventory management and supply chain optimization.

This project addresses the challenges of demand uncertainty, helping businesses reduce stockouts and overstocking. The results will be utilized by supply chain managers and analysts for data-driven decision-making.

The data will consist of time-stamped sales records, including demand quantities, prices, and promotional events. Stored in tabular formats (e.g., CSV), the dataset will span multiple months or years, structured  as a matrix  for time periods and demand features.

Quality will be assessed using performance metrics such as RMSE and MAPE. The final report will include a summary of these metrics, visual comparisons of forecast accuracy, sensitivity analysis of model parameters, and an evaluation of potential improvements.

Feasibility will be demonstrated through pilot studies and performance baselines. Risks include data quality issues  and unforeseen market shifts. 

Key conditions include data availability (1–2 years of historical data), data cleanliness, access to analytical tools (e.g., Python, R), and collaboration among data scientists, domain experts, and supply chain managers to align the model with business needs.

The project will explore various forecasting methods, including ARIMA, ETS, Random Forest, and LSTM neural networks. Key hypotheses involve the dependence of demand on historical data and external factors, with a focus on identifying optimal probability models based on their predictive performance and robustness.


\section{Research or development?}
The model developed in this research is expected to remain useful for several years, provided it is regularly updated and adapted to new data and evolving market conditions. In its initial phase, the model will be applied to forecast customer demand based on historical sales and external factors, helping businesses optimize inventory, production schedules, and supply chain logistics. Over time, the model can be fine-tuned to account for changes in consumer behavior, new product lines, and emerging trends, ensuring its continued relevance. Its ability to integrate new variables and adapt to different business contexts will extend its lifecycle, making it a valuable tool for businesses seeking to maintain a competitive edge in forecasting accuracy.

In the future, the model may be replaced or augmented by more advanced techniques, such as reinforcement learning, which can adapt in real-time to changing market conditions. As artificial intelligence and deep learning technologies evolve, we may see the development of even more sophisticated demand forecasting models capable of making autonomous decisions and predicting demand with greater precision. These future models may leverage larger, more diverse datasets, including unstructured data like social media sentiment or real-time consumer interactions, to enhance predictive power. However, the core principles and methods derived from this research will likely continue to influence future developments in demand forecasting, serving as a foundation for next-generation models.

%\bibliographystyle{unsrt}
%\bibliography{Name-theArt}
\end{document}